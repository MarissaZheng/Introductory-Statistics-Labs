\documentclass[]{article}
\usepackage{lmodern}
\usepackage{amssymb,amsmath}
\usepackage{ifxetex,ifluatex}
\usepackage{fixltx2e} % provides \textsubscript
\ifnum 0\ifxetex 1\fi\ifluatex 1\fi=0 % if pdftex
  \usepackage[T1]{fontenc}
  \usepackage[utf8]{inputenc}
\else % if luatex or xelatex
  \ifxetex
    \usepackage{mathspec}
  \else
    \usepackage{fontspec}
  \fi
  \defaultfontfeatures{Ligatures=TeX,Scale=MatchLowercase}
\fi
% use upquote if available, for straight quotes in verbatim environments
\IfFileExists{upquote.sty}{\usepackage{upquote}}{}
% use microtype if available
\IfFileExists{microtype.sty}{%
\usepackage{microtype}
\UseMicrotypeSet[protrusion]{basicmath} % disable protrusion for tt fonts
}{}
\usepackage[margin=1in]{geometry}
\usepackage{hyperref}
\hypersetup{unicode=true,
            pdftitle={Statistics 218: Labs Outline},
            pdfborder={0 0 0},
            breaklinks=true}
\urlstyle{same}  % don't use monospace font for urls
\usepackage{graphicx,grffile}
\makeatletter
\def\maxwidth{\ifdim\Gin@nat@width>\linewidth\linewidth\else\Gin@nat@width\fi}
\def\maxheight{\ifdim\Gin@nat@height>\textheight\textheight\else\Gin@nat@height\fi}
\makeatother
% Scale images if necessary, so that they will not overflow the page
% margins by default, and it is still possible to overwrite the defaults
% using explicit options in \includegraphics[width, height, ...]{}
\setkeys{Gin}{width=\maxwidth,height=\maxheight,keepaspectratio}
\IfFileExists{parskip.sty}{%
\usepackage{parskip}
}{% else
\setlength{\parindent}{0pt}
\setlength{\parskip}{6pt plus 2pt minus 1pt}
}
\setlength{\emergencystretch}{3em}  % prevent overfull lines
\providecommand{\tightlist}{%
  \setlength{\itemsep}{0pt}\setlength{\parskip}{0pt}}
\setcounter{secnumdepth}{0}
% Redefines (sub)paragraphs to behave more like sections
\ifx\paragraph\undefined\else
\let\oldparagraph\paragraph
\renewcommand{\paragraph}[1]{\oldparagraph{#1}\mbox{}}
\fi
\ifx\subparagraph\undefined\else
\let\oldsubparagraph\subparagraph
\renewcommand{\subparagraph}[1]{\oldsubparagraph{#1}\mbox{}}
\fi

%%% Use protect on footnotes to avoid problems with footnotes in titles
\let\rmarkdownfootnote\footnote%
\def\footnote{\protect\rmarkdownfootnote}

%%% Change title format to be more compact
\usepackage{titling}

% Create subtitle command for use in maketitle
\newcommand{\subtitle}[1]{
  \posttitle{
    \begin{center}\large#1\end{center}
    }
}

\setlength{\droptitle}{-2em}

  \title{Statistics 218: Labs Outline}
    \pretitle{\vspace{\droptitle}\centering\huge}
  \posttitle{\par}
    \author{}
    \preauthor{}\postauthor{}
    \date{}
    \predate{}\postdate{}
  

\begin{document}
\maketitle

\section{Lab 1: What is data?}\label{lab-1-what-is-data}

\subsection{Topics:}\label{topics}

\begin{itemize}
\tightlist
\item
  Describing variables in a dataset
\item
  Interpreting observations in real-world terms
\item
  Calculating numerical summaries of samples
\item
  Deciding which variables to collect or analyze
\item
  Discussing evidence for specific research questions
\end{itemize}

\subsection{Dataset: Titanic (abbreviated
version)}\label{dataset-titanic-abbreviated-version}

\subsection{R Skills:}\label{r-skills}

\begin{itemize}
\tightlist
\item
  Getting used to R Markdown ⦁* Reading a dataset into R ⦁* Looking at,
  and interpreting, the\texttt{summary()} of the dataset ⦁* Looking at,
  and interpreting, individual rows of the dataset:\texttt{head()},
  \texttt{{[}1,{]}} ⦁* Checking for obvious errors and missing data:
  \texttt{na.omit()} ⦁* Filtering to look at specific observation and
  variables:\texttt{filter()}, \texttt{select()} ⦁* Mutating variables
  by factoring, combining, etc: \texttt{mutate()},
  \texttt{mutate\_at()}, \texttt{factor()} ⦁* Calculating summaries of
  samples, including by group: \texttt{mean()}, etc;
  \texttt{summarize\_at()}, \texttt{group\_by()}
\end{itemize}

\subsection{Assignment: Full Titanic dataset: who lived and
died?}\label{assignment-full-titanic-dataset-who-lived-and-died}

\subsection{Additional Resources:}\label{additional-resources}

\begin{itemize}
\tightlist
\item
  DataCamp
\item
  RStudio Cheatsheets: dplyr
\item
  Swirl
\end{itemize}

\section{Lab 2: Visualization}\label{lab-2-visualization}

\subsection{Topics:}\label{topics-1}

Reading, interpreting, comparing, and knowing when to use\ldots{} *
Histograms * Dot plots * Box plots * Bar graphs * {[}Pie Charts -
omitted in R section{]} * Side-by-Side box plots * Side-by-Side bar
graphs * Stacked bar graphs * Scatter plots * Line plots

\subsection{Dataset: Titanic
(abbreviated)}\label{dataset-titanic-abbreviated}

\subsection{R Skills:}\label{r-skills-1}

\begin{itemize}
\tightlist
\item
  Creating all the above in ggplot
\item
  Grouping and facetting
\item
  Optional bells and whistles for plotting
\end{itemize}

\subsection{Assignment: Titanic final
report}\label{assignment-titanic-final-report}

\subsection{Additional Resources:}\label{additional-resources-1}

\begin{itemize}
\tightlist
\item
  RStudio Cheatsheets: ggplot2
\item
  Colors pdf
\item
  Link some tutorials?
\end{itemize}

\section{Lab 3: Random Variables, Categorical/discrete
distributions}\label{lab-3-random-variables-categoricaldiscrete-distributions}

\subsection{Topics:}\label{topics-2}

\begin{itemize}
\tightlist
\item
  Random variables: samples and populations, statistics and parameters
\item
  Frequency tables
\item
  Probability
\item
  The Binomial Distribution
\item
  Quantifying evidence: baby chi-square?
\end{itemize}

\subsection{Dataset: Election data?}\label{dataset-election-data}

\subsection{Skills:}\label{skills}

\begin{itemize}
\tightlist
\item
  {[}review{]} Counting up categoricals
\item
  Making a frequency table from data: \texttt{reshape} stuff
\item
  Using \texttt{pbinom()} and \texttt{qbinom()}
\item
  Simulation? \texttt{rbinom()}
\end{itemize}

\subsection{Assignment: Hmmmm\ldots{}..}\label{assignment-hmmmm..}

\subsection{Additional Resources}\label{additional-resources-2}

\begin{itemize}
\tightlist
\item
  Applets!
\end{itemize}

\section{Lab 4: Hypothesis Testing}\label{lab-4-hypothesis-testing}

\subsection{Topics:}\label{topics-3}

\emph{⦁ Hypothesis test general principles } 1-sample prop test * very
gentle intro to confidence interval for \(p\) * Chi-Square Test

\subsection{Dataset: same as lab 3 ideally. Election
data?}\label{dataset-same-as-lab-3-ideally.-election-data}

\subsection{R Skills:}\label{r-skills-2}

\begin{itemize}
\tightlist
\item
  {[}review{]} making frequency tables
\item
  {[}review{]} \texttt{pbinom()}, \texttt{qbinom()}
\item
  \texttt{pchisq()}, \texttt{qchisq()}, maybe? probably.
\item
  \texttt{prop.test()}
\item
  \texttt{chisq.test()}
\end{itemize}

\subsection{Assignment: Final report on
{[}biodata{]}.}\label{assignment-final-report-on-biodata.}

\subsection{Additional Resources:}\label{additional-resources-3}

\begin{itemize}
\tightlist
\item
  Applets
\item
  Real-world chi-square study abstracts and such
\end{itemize}

\section{Lab 5: Densities, the CLT}\label{lab-5-densities-the-clt}

\subsection{Topics:}\label{topics-4}

\begin{itemize}
\tightlist
\item
  Density curves
\item
  The Uniform distribution
\item
  The Normal distribution
\item
  CLT
\item
  Normal approximation to Binomial
\end{itemize}

\subsection{Dataset: basketball??? hmmmmm I don't love
it\ldots{}.}\label{dataset-basketball-hmmmmm-i-dont-love-it.}

\subsection{R Skills:}\label{r-skills-3}

\begin{itemize}
\tightlist
\item
  \texttt{runif()}, \texttt{punif()}, \texttt{qunif()}
\item
  \texttt{rnorm()}, \texttt{dnorm()}, \texttt{qnorm()}, \texttt{pnorm()}
\item
  Curves in \texttt{ggplot}
\item
  simulation\ldots{}.?
\item
  qq plots
\end{itemize}

\subsection{Assignment: Not bball. Make them do a different
dataset.}\label{assignment-not-bball.-make-them-do-a-different-dataset.}

\section{Lab 6: t-tests and confidence
intervals}\label{lab-6-t-tests-and-confidence-intervals}

\subsection{Topics:}\label{topics-5}

\begin{itemize}
\tightlist
\item
  one-sample t-test
\item
  two-sample t-test
\item
  Confidence intervals for \(\mu\) and \(\mu_1 - \mu_2\)
\end{itemize}

\subsection{Dataset: wine?}\label{dataset-wine}

\subsection{R Skills:}\label{r-skills-4}

\begin{itemize}
\tightlist
\item
  \texttt{t.test()}
\item
  Maybe conf int calculations?
\end{itemize}

\subsection{Assignment: wine? options?}\label{assignment-wine-options}

\subsection{Additional Resources:}\label{additional-resources-4}

\begin{itemize}
\tightlist
\item
  Real world abstracts
\end{itemize}

\section{Lab 7: ANOVA}\label{lab-7-anova}

\subsection{Topics:}\label{topics-6}

\begin{itemize}
\tightlist
\item
  ANOVA
\item
  Multiple testing, Tukey HSD
\end{itemize}

\subsection{Dataset: bodwins}\label{dataset-bodwins}

\subsection{R Skills:}\label{r-skills-5}

\begin{itemize}
\tightlist
\item
  {[}review{]} Side-by-side boxplots
\item
  \texttt{anova()}, \texttt{lm()}, \texttt{aov()}
\item
  \texttt{tukeyHSD()}
\end{itemize}

\subsection{Assignment: your family or
similar}\label{assignment-your-family-or-similar}

\subsection{Additional Resources:}\label{additional-resources-5}

\begin{itemize}
\tightlist
\item
  Real world abstracts
\end{itemize}

\section{Lab 8: Regression}\label{lab-8-regression}

\subsection{Topics:}\label{topics-7}

\begin{itemize}
\tightlist
\item
  Linear models
\item
  Least squares and residuals
\item
  Residual plots
\item
  Transformation of variables?
\end{itemize}

\subsection{Dataset: kellys}\label{dataset-kellys}

\subsection{R Skills:}\label{r-skills-6}

\begin{itemize}
\tightlist
\item
  \texttt{lm()}
\item
  {[}review{]} scatterplots
\item
  Adding model to scatterplot
\item
  Prediction
\item
  Plotting residuals (hmmm why does it suck in ggplot. maybe worth
  building by hand)
\end{itemize}

\subsection{Assignment: your choice of names
stuff}\label{assignment-your-choice-of-names-stuff}

\subsection{Additional Resources}\label{additional-resources-6}

\section{Lab 9: Final project}\label{lab-9-final-project}


\end{document}
